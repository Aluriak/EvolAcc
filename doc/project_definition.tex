%%%%%%%%%%%%%%%%%%%%%%%%%
% PACKAGES              %
%%%%%%%%%%%%%%%%%%%%%%%%%
\documentclass{report} % book|article|…

\usepackage[utf8x]{inputenc}    % accents
\usepackage{geometry}           % marges
\usepackage[english]{babel}     % langue
\usepackage{graphicx}           % images
\usepackage{verbatim}           % texte préformaté
\usepackage{fancyhdr}           % fancy
\usepackage{filecontents}       % write file directly
\usepackage{listings}           % source code 
\usepackage{url}                % clickable urls 


%%%%%%%%%%%%%%%%%%%%%%%%%
% PRÉAMBULE             %
%%%%%%%%%%%%%%%%%%%%%%%%%
\title{EvolAcc Project Definition} 
\author{Lucas BOURNEUF}
% empty for compilation date
\date{} 

% FORMAT PAGES         
\pagestyle{fancy} 
        \lhead{} % left head
        \chead{EvolAcc Mendel} % center head
        \rhead{} % right head
        \lfoot{} % left foot
        \cfoot{\thepage/\pageref{LastPage}} % center foot
        \rfoot{} % right foot






%%%%%%%%%%%%%%%%%%%%%%%%%
% BEGIN                 %
%%%%%%%%%%%%%%%%%%%%%%%%%
\begin{document}

%%%%%%%%%%%%%%%%%%%%%%%%%
% SECTION               %
\section*{First words}
    \paragraph*{}
    This document describes some ideas about \textbf{EvolAcc} project of life simulator, designed 
    for be a sandbox for artificial life. What is life, 
    what are genes,... All theses definitions belong to next part.
    \paragraph*{}
    EvolAcc have a strong emphasis on modularity : the code is heavily modular,
    and designed for terminal or final GUI users, and is also designed for 
    integration inside Python projects, just like a Python module.
    \paragraph*{}
    Tools involved : 
    \begin{description}
            \item[Python 3] for main code; 
            \item[C/C++] for optimizations;
            \item[Qt] for GUI; 
            \item[ncurses] for terminal GUI;
            \item[git] for version managing, through \url{github.com};
            \item[pew] for environnement management;
            \item[graphviz] for environnement management;
    \end{description}
    Python 3 is mainly battery included, but some \textit{pip installable} modules can be necessary. 
    (notabily for argument parsing, automatic JIT compilation,...)



%%%%%%%%%%%%%%%%%%%%%%%%%
% SECTION               %
\section*{Units}
    \paragraph*{}
    Unit object is defined as a Proxy to componants that are stocked inside Unit itself.
    Mutate, apply a phase (\textit{step}),… are operations asked to Units, which are 
    relays them to their componants.
    \paragraph*{}
    Componants are Genome, Property and Quantity instances, contained in a list.
    Units with Genomes are considered like Life.
    A Unit can have zero componants. 
    \paragraph*{}
    Each Unit can be seen as an object in an Environnement.
    When a Unit is asked to step, it receive Environment as a context of runtime, 
    and must return a boolean value, which is True iff Unit must be removed from
    simulation. (mainly used by Life, for simulate dying)



%%%%%%%%%%%%%%%%%%%%%%%%%
% SECTION               %
\section*{Environment}



%%%%%%%%%%%%%%%%%%%%%%%%%
% SECTION               %
\section*{Genome and Genomes}




%%%%%%%%%%%%%%%%%%%%%%%%%
% SECTION               %
    \section*{Objectives for version \textbf{0.1.0}}
    \paragraph*{}
    \begin{description}
        \item[Primary tools] must be fonctionnal, notabily for UML and documentation production;
        \item[Arch] of modules and git repository must be created;
        \item[Simple] Environment with very simple and useless life should be runnable;
        \item[Observer] of Environment that simply show in console a view of Life;
    \end{description}




%%%%%%%%%%%%%%%%%%%%%%%%%
% SECTION               %
    \section*{Objectives for version \textbf{0.2.0}}
    \paragraph*{}
    %\begin{description}
        %\item[] ;
    %\end{description}




%%%%%%%%%%%%%%%%%%%%%%%%%
% SECTION               %
\section*{Improvements for next versions}
    \paragraph*{}
    Main goals of next versions. Feasibility is considered OK and next updates can bring some of theses improvements.
    \begin{description}
            \item[Genome] integration of both Static and Dynamic Genome;
            \item[Property] for allow definitions of environmental pressure and gradient of parameter;
    \end{description}

    \paragraph*{}
    Planned functionnality. Maybe in a long time, but its something we want to see.
    \begin{description}
            \item[Quantity]: each unit of Life have an amount of each elements (Energy, sugar,...) and can, by use of function, collect, transform and drop elements in Environnement; 
            \item[Transposons]: a not translated word of vocabulary in DNA can be recognize by special objects, like plasmids, that inserts themselves inside the sequence;
                    In result, we have modifications of genomes and transmission of plasmids;
            \item[Viruses] like transposons but for both Dynamic and Static Genome. Simply consider adding of Genome componant to already Genome possessor. Stranger DNA can be added to any life;
            \item[DataSystem] EvolAcc Observer, designed for save and load Environment, Life configuration,… ;
            \item[Phylogenient] EvolAcc Observer that creat a phylogenetic tree according to real evenments;
            \item[Phylogenius] EvolAcc Observer that creat a phylogenetic tree according to final data, with bioinformatics methods;
            \item[GUI] for final users, in Qt and/or ncurses;
    \end{description}

    \paragraph*{}
    Some ideas that need to be studied, not expected before many updates.
    \begin{description}
            \item[Telemetrox] EvolAcc Observer that print data of current step in a webpage; 
    \end{description}




%%%%%%%%%%%%%%%%%%%%%%%%%
% SECTION               %
\section*{Tools}
    \paragraph*{}
    Lots of modules, libraries,… can be used for observers.
    This section will certainly be quickly updated.
    

    

%%%%%%%%%%%%%%%%%%%%%%%%%
% SECTION               %
\section*{Versionning}
    \paragraph*{}
    Use of SemVer versionning naming.\\
    \textbf{EvolAcc} is the name of the project and of the final program.
    Each major version will have a codename, taked from a famous biologist.\\
    Moreover, each minor version will have a codename, 
    taked from the Periodic Table of elements.
    



%%%%%%%%%%%%%%%%%%%%%%%%%
% SECTION               %
\section*{Coding conventions}
    \paragraph*{}
    All python code must follow the PEP 8 recommendations\footnote{\url{https://www.python.org/dev/peps/pep-0008/}}, 
    as possible, and according to the case, some deviations are allowed.
    Objectives of conventions are \textbf{code clarity and uniformization}, not a PEP 8 inconditionnal respect. 
    \paragraph*{}
    Important points of coding conventions :
    \begin{itemize}
        \item 4 spaces by indentation;
        \item snake case for methods, functions, variables and modules;
        \item camel case for class names;
        \item upper case for constants;
    \end{itemize}
    



%%%%%%%%%%%%%%%%%%%%%%%%%
% END                   %
%%%%%%%%%%%%%%%%%%%%%%%%%
\end{document}
% END



