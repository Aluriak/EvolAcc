%%%%%%%%%%%%%%%%%%%%%%%%%
% PACKAGES              %
%%%%%%%%%%%%%%%%%%%%%%%%%
\documentclass{report} % book|article|…

\usepackage[utf8x]{inputenc}    % accents
\usepackage{geometry}           % marges
\usepackage[english]{babel}     % langue
\usepackage{graphicx}           % images
\usepackage{verbatim}           % texte préformaté
\usepackage{fancyhdr}           % fancy
\usepackage{filecontents}       % write file directly
\usepackage{listings}           % source code 
\usepackage{url}                % clickable urls 


%%%%%%%%%%%%%%%%%%%%%%%%%
% PRÉAMBULE             %
%%%%%%%%%%%%%%%%%%%%%%%%%
\title{Project Definition} 
\author{Lucas BOURNEUF}
% empty for compilation date
\date{} 

% FORMAT PAGES         
\pagestyle{fancy} 
        \lhead{} % left head
        \chead{} % center head
        \rhead{} % right head
        \lfoot{} % left foot
        \cfoot{\thepage/\pageref{LastPage}} % center foot
        \rfoot{} % right foot






%%%%%%%%%%%%%%%%%%%%%%%%%
% BEGIN                 %
%%%%%%%%%%%%%%%%%%%%%%%%%
\begin{document}

%%%%%%%%%%%%%%%%%%%%%%%%%
% SECTION               %
\section*{First words}
    \paragraph*{}
    This document describes some ideas about a project of life simulator, designed 
    for be a sandbox for artificial life. What is life, what are genes,... All theses definitions belong to next part.
    \paragraph*{}
    Tools involved : 
    \begin{description}
            \item[Python 3] for main code; 
            \item[C/C++] for optimizations;
            \item[Qt] for GUI; 
            \item[ncurses] for terminal GUI;
            \item[git] for version managing, through \url{github.com};
            \item[pew] for environnement management;
    \end{description}
    Python 3 is mainly battery included, but some \textit{pip installable} modules can be necessary. 
    (notabily for argument parsing, automatic JIT compilation,...)



%%%%%%%%%%%%%%%%%%%%%%%%%
% SECTION               %
\section*{Basic definitions}
    \paragraph*{}
    Each particle of Life can be seen as an object manipulate by an Environnement, defined by its Properties.
    Environnement' space is divided in Square, and Environnement's time in phase.
    A particle of Life, or just Life, have a knowledge of immediate neighbors\footnote{Moore neighbors, by example}.

    \paragraph*{}
    Life is defined mainly by a DNA sequence, composed of 0 and 1, or ATGC eventually. 
    This DNA contains all information the Life will need for know what to do at each phase.
    \begin{itemize}
            \item definition of values (like maximum of amount of an element, see Add-ons part); 
            \item behavior (with the help of a DNA compiler, see Tools part);  
            \item DNA-linked values (see Transposons in Add-ons part); 
    \end{itemize}

    \paragraph*{}
    Environnement is defined by a space divided in Square. Each of them can contains one or zero Life particle, 
    and have some properties.
    Property object can provides values like temperature, pression, current,...
    These properties, when evolved, modify Natural Selection.



%%%%%%%%%%%%%%%%%%%%%%%%%
% SECTION               %
\section*{Add-ons}
    \paragraph*{}
    Some ideas that need to be studied, when main parts will be done.
    \begin{description}
            \item[Elements] each unit of Life have an amount of each elements (Energy, sugar,...) and can, by use of function, collect, transform and drop elements in Environnement; 
            \item[Transposons] a not translated word of vocabulary in DNA can be recognize by special objects, like plasmids, that inserts themselves inside the sequence. 
                    In result, we have modifications of genomes and transmission of plasmids.
    \end{description}



%%%%%%%%%%%%%%%%%%%%%%%%%
% SECTION               %
\section*{Tools}
    \paragraph*{}
    A very important tool is a DNA compiler, that transform DNA in fully valid Python source code, that can be validated and used.
    Compilation of a Life particle DNA is performed at the particle creation, and when DNA change (see Transposons in Add-ons part).
    \begin{lstlisting}[language=Python]
        cc = DNACompiler(
            alphabet='ab', 
            vocabulary=['ka', 'me', 'ha']
        )
        # tables associate lexems to vocabulary
        print(cc.tables) 
        # >>>> {'aa':'ka', 'ab':'me', 'ba':ha', 'bb':''}
        print(cc.compile('aaabbaabba')
        # >>>> 'kamehameha'
    \end{lstlisting}
    This approach permit many things :
    \begin{itemize}
        \item production of Python code by simply defined the right templates and vocabulary;
        \item genetic code redundancy because many lexems can generate the same values;
        \item complexe behavior generation lexems can generate boolean operators, by examples;
        \item garanted correctness when lexems lead to invalid code, it can be ignored;
    \end{itemize}
    \newpage

    \paragraph*{}
    Many functions can be provided, like access to genetic distance of a neighbor, Environnement's properties,...
    Finally, complexes and adapted codes will appear : mutation of the DNA lead to modifications of the codes.

    \paragraph*{}
    For provide behavior, structure of the code can be always the same, and correspond to the regex [CA]*, where C are condition, and A an action.
    By example, CCACACCAA is equivalent to :
    \begin{lstlisting}[language=Python]
        if cond1 and cond2:
            action1()
        if cond3:
            action2()
        if cond4 and cond5:
            action3()
            action4()
    \end{lstlisting}
    CCA is readed in DNA, and finally transcripted in Python code. 
    More complete example :
    \begin{lstlisting}[language=Python]
        actions    = ['self.die']
        conditions = ['world.temperature', 'self.age', 
                      'self.resistance', '>']
        cc = DNACompiler(alphabet='01', 
                vocabulary\_actions=actions, 
                vocabulary\_conditions=conditions
        )
        print(cc.tables)
        # >>>> {
                '000':('C', 'world.temperature'), 
                '001':('C', 'self.age'), 
                '010':('C', 'self.resistance'), 
                '011':('A', 'self.die'), 
                '100':('C', '>'),
                '101':('V', ''), # value in the next lexem
                '110':('?', '')
                '111':('?', '')
        }
        print(cc.compile('000100101011001100101111011')
        # >>>> """
        if world.temperature() > self.resistance():
            self.die()
        if self.age() > 7: 
            self.die()
        """
    \end{lstlisting}
    Here, value is taked by read the next three bits (here, '111' so 7 in base 2). But it can be obtain by reread all DNA, by block of 8 bits, where each block is a value.
    

    

%%%%%%%%%%%%%%%%%%%%%%%%%
% SECTION               %
\section*{Project Names}
    \paragraph*{}
    Lucas vote for a name like \textit{EvolAcc} with a subname like \textit{Argon}, \textit{Carbon}, or another element of the periodic table.


%%%%%%%%%%%%%%%%%%%%%%%%%
% END                   %
%%%%%%%%%%%%%%%%%%%%%%%%%
\end{document}
% END



